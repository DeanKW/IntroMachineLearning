  %%%%%%%%%%%%%%%%%%%%%%%%%%%%%%%%%%%%%%%%%%%%%%%%%%%%%%%%%%%%%%%%%%%%%%%%%%%%%%%%%%%%
% Do not alter this block (unless you're familiar with LaTeX
\documentclass{article}
\usepackage[margin=1in]{geometry}
\usepackage[dvipsnames]{xcolor}
\usepackage{tcolorbox}
\usepackage{float}
\usepackage{amsmath,amsthm,amssymb,amsfonts, fancyhdr, color, comment, graphicx, environ,mathtools}
\usepackage[parfill]{parskip}
\usepackage{mdframed}
\usepackage[shortlabels]{enumitem}
\usepackage{indentfirst}
\usepackage{hyperref}


\hypersetup{
	colorlinks=true,
	linkcolor=blue,
	filecolor=magenta,      
	urlcolor=blue,
}

\newcommand\aug{\fboxsep=-\fboxrule\!\!\!\fbox{\strut}\!\!\!}


\pagestyle{fancy}


\newenvironment{problem}[2][Problem]
{ \begin{mdframed}[backgroundcolor=gray!20] \textbf{#1 #2} \\}
	{  \end{mdframed}}

% Define solution environment
\newenvironment{solution}
{\textit{Answer:}}
{}

%%%%%%%%%%%%
% NOTE
%%%%%%%%%%%%
\makeatletter
\newfloat{info@box}{tbp}{loi}[section]% 1: Name of float environment. 2: Default placement (top, bottom, ...). 3: File extension if written to an aux-file (like toc, lof, lot, loa, ...). 4: Numbering within <section/subsection/...>.
\makeatother
\floatname{info@box}{Infobox}% Adapt caption.

\newenvironment{infobox}[1][]{% Create new environment using info@box and tcolorbox
	%\begin{info@box}%
		\begin{tcolorbox}[colback=red!15!white,%                    background color
			colframe=red!75!black,%                               frame color
			title=Additional information\ifstrempty{#1}{}{: #1}.% title
			]%
		}{%
		\end{tcolorbox}%
%	\end{info@box}%
}


\newenvironment{note}[1][]{% Create new environment using info@box and tcolorbox
%	\begin{info@box}%
		\begin{tcolorbox}[colback=blue!15!white,%                    background color
			colframe=blue!75!black,%                               frame color
			title=Note\ifstrempty{#1}{}{: #1}% title
			]%
		}{%
		\end{tcolorbox}%
	%\end{info@box}%
}


\newenvironment{tip}[1][]{% Create new environment using info@box and tcolorbox
	\begin{info@box}%
		\begin{tcolorbox}[colback=green!15!white,%                    background color
			colframe=green!75!black,%                               frame color
			title=Tip!\ifstrempty{#1}{}{: #1},% title
			rightrule=300mm]%
		}{%
		\end{tcolorbox}%
	\end{info@box}%
}

\newenvironment{aside}[1][]{% Create new environment using info@box and tcolorbox
	%	\begin{info@box}%
		\begin{tcolorbox}[colback=yellow!15!white,%                    background color
			colframe=yellow!60!black,%                               frame color
			title=An Aside\ifstrempty{#1}{}{: #1},% title
			]%
		}{%
		\end{tcolorbox}%
		%\end{info@box}%
	}
%%%%%%%%%%%%
% END NOTE
%%%%%%%%%%%%

\renewcommand{\qed}{\quad\qedsymbol}

\DeclareMathOperator*{\argmin}{argmin} 
\DeclareMathOperator*{\argmax}{argmin}

\newtheorem*{bookequation}{Equation}

% prevent line break in inline mode
\binoppenalty=\maxdimen
\relpenalty=\maxdimen

\date{\vspace{-5ex}}
\lhead{github.com/DeanKW/IntroMachineLearning}
\rhead{Chapter 3: Linear Regression}
\chead{\textbf{ISLP Notes}}
\begin{document}
    \section*{Chapter 3 Summary: Linear Regression}\label{chapter-3-summary-linear-regression}
        Linear regression (yes, the one from middle or high school) may seem boring and not worth your time, but it is! 
        It is widely used and many    fancier models are generalizations or extensions of linear regression.    So seriously, this is important.
        \subsection*{3.1 Simple Linear Regression}
        \textbf{Simple Linear Regression} assumes that the response variable, \(Y\) has a linear relationship to a single predictor variable \(X\).
        \begin{equation*}\tag{3.1}
            Y \approx \beta_0 + \beta_1 X
        \end{equation*}
        
        
        \begin{note}
            This is the same relationship as \(y=mx+b\) you likely remember from school.\\
            \(\beta_0\) represents the intercept, \(b\) while \(\beta_1\) represents the slope, \(m\).
        \end{note}
        
        \(\beta_0\) and \(\beta_1\) are referred to as the \textbf{coefficients} or \textbf{parameters}.\\
        We will use our training data to estimate these parameters and then indicate and reprsent our \textbf{model} as
        \begin{equation*}\tag{3.2}
            \hat y = \hat \beta_0 + \hat \beta_1 X
        \end{equation*}

        \begin{note}
            We call this ``hat notation.'' In statistics, a hat over something indicates it's either an
            \href{https://en.wikipedia.org/wiki/Estimator}{estimator} or an estimated value.
        \end{note}
        
        \subsubsection*{3.1.1  Estimating the Coefficients}\label{estimating-the-coefficients}
        
        Since we do not know the true relationship between \(X\) and \(Y\), we use the training data to estimate \(\beta_0\) and \(\beta_1\)

        \begin{note}
            Just like you did in middle/high school when calculating the line of best fit
        \end{note}
        
        We will use \textbf{least squares} as the criteria to determine what values to use for our parameters, \(\beta_0\) and \(\beta_1\).
        
        The \(i\)th \textbf{residual} is the difference between the \(i\)th response variable and our prediction for that variable
        \begin{equation*}
        e_i = y_i - \hat y_i
        \end{equation*}
        
        The \textbf{residual sum of squares} is defined
        
        
        \begin{align*}
            \text{RSS} &= e_1^2 + e_2^2 + \cdots + e_n^2 \\
            &= (y_1 - \hat y_1)^2 +  (y_2 - \hat y_2)^2 + \cdots +  (y_n- \hat y_n)^2 \\
            &= (y_1 - \hat \beta_0 - \hat \beta_1 x_1)^2 + (y_2 - \hat \beta_0 - \hat \beta_1 x_2)^2  + \cdots + (y_n - \hat \beta_0 - \hat \beta_1 x_n)^2 
        \end{align*}
        
        
        The parameters can be calculated as:
        
        \begin{align*}\tag{3.4}
        \hat \beta_1 &= \frac{\sum_{i=1}^n(x_i - \bar x)(y_i-\bar y)}{\sum_{i=1}^n(x_i - \bar xx)^2}\\
        \hat \beta_0 &= \bar y - \hat \beta_1 \bar x
    \end{align*}
     where \(\bar y\) and \(\bar x\) are the sample mean, defined below
     \begin{equation*}
           \bar y = \frac{1}{n}\sum_{i=1}^n y_i \qquad \text{and} \qquad \bar x = \frac{1}{n}\sum_{i=1}^n x_i
     \end{equation*}

        \subsubsection*{3.1.2 Assessing the Accuracy of the Coefficient Estimates}
        \paragraph{Standard Error\\}
        \textbf{Standard error} tells us the average amount that an estimate differs from the actual value.
        
        \begin{equation*} \tag{3.7}
            \text{Var}(\hat \mu) = \text{SE}(\hat \mu)^2 = \frac{\sigma^2}{n}
        \end{equation*}
        
        
        \begin{equation*} \tag{3.8}
            \text{SE}(\hat \beta_0)^2 = \sigma^2 \left[ \frac{1}{n} + \frac{\bar x ^2}{\sum_{i=1}^n (x_i - \bar x)^2} \right]
        \end{equation*}
        
        \begin{equation*} \tag{also 3.8}
            \text{SE}(\hat \beta_1)^2 = \frac{\sigma^2}{\sum_{i=1}^n (x_i - \bar x)^2}
        \end{equation*}
        
        where \(\sigma^2 = \text{Var}(\epsilon)\) While we generally don't know \(\sigma^2\), it can be estimated from the data.\\
        The estimate of \(\sigma\) is called \textbf{residual standard error}
        \begin{equation*}
            \sigma = \text{RSE} = \sqrt{\frac{\text{RSS}}{n-2}}
        \end{equation*}
        
        \paragraph{Confidence Intervals\\}
         Standard errors can be used to compute    \textbf{confidence intervals}. The 95\% confidence interval for
        \(\beta_0\) and \(\beta_1\) are approximately: 
        \begin{gather*}
            \hat \beta_0 \pm 2 \cdot \text{SE}(\hat \beta_0) \tag{3.11}\\
            \hat \beta_1 \pm 2 \cdot \text{SE}(\hat \beta_1)
        \end{gather*}
        
        This means there is approximately a 95\% chance that the interval
        \begin{equation*} \tag{almost 3.10}
            \left[\beta_0 - 2 \cdot \text{SE}(\hat \beta_0), \beta_0 + 2 \cdot \text{SE}(\hat \beta_0)\right]            
        \end{equation*}
        
        contains the true value for \(\beta_0\). 
        \textbf{NOTE: This is an incorrect interpretation of a confidence interval, but it is what the book writes.  Please see the warning below}
        
        \begin{note}
            1.96 is closer to the correct value than 2. This value comes from the Z-score value for a 97.5\% quantile of a t-distribution. You
            are likely to see these in any statistics class.
        \end{note}
        
        \begin{warning}
            A 95\% confidence interval does \textbf{NOT} mean we there is a 95\% chance that the true parameter lies within the range.\\
            What it really means is that if we sampled the data 100 times, each time calculating the parameter and confidence interval, 95\% of those
            confidence intervals would contain the true value of the parameter.\\
            It's a small distinction, but I wanted to make it, even if the book did not.\\
            We generally say that we are 95\% confident that the true parameter lies in the range, not that the probability is 0.95.
        \end{warning}
        
        \paragraph{Hypothesis Testing\\}
        Standard errors can also be used to perform hypothesis testing.  The most common of which is the \emph{null hypothesis}.  We will test to see if
        the data provides evidence to reject the null hypothesis (that two variables/phenomena/results have no relationship) in favor of the alternative hypothesis (that there is a relationship)
        \begin{equation*}\tag{3.12}
            \underbrace{H_0}_{\text{Null hypothesis}}: \text{There is no relationship between } X \text{ and } Y \Longrightarrow \beta_1=0
         \end{equation*}

         \begin{equation*}\tag{3.13}
            \underbrace{H_1}_{\text{Alternative hypothesis}}: \text{There is a relationship between } X \text{ and } Y\Longrightarrow \beta_1 \neq 0
         \end{equation*}
         We compute the \textbf{t-statistic}:
         \begin{equation*}\tag{3.14}
             t = \frac{\hat \beta_1-0}{\text{SE}(\hat \beta_1)}
          \end{equation*}
          If $\beta_1 = 0$, this will have a $t$-distribution with $n-2$ degrees of freedom.  The probability of observing a value greater than or equal to $|t|$ is called a \textbf{p-value} .  If the p-value is small, there it is unlikely that the relationship between predictor and response is due to chance.
         We \textbf{reject} the null hypothesis (and accept the alternative hypothesis, if the p-value is "small enough")
         
         \begin{interesting}
             Did you know that the $t$-test was developed by in order to make better beer?  William Sealy Gosset, while the head experimental brewer at Guinness, developed the $t$-test as a way to study the quality of the barley used in brewing Guinness.
         \end{interesting}
         
         \textbf{DEAN, ADD MORE}
         
        \subsubsection*{3.1.3 Assessing the Accuracy of the Model}
        \paragraph{Residual Standard Error\\}
        Suppose we knew the exact true model, $Y\approx \beta_0+\beta_1 X$ (from equation 3.1), recall that there is the irreducible error, $\epsilon$, associated with every term.  Residual Standard Error (RSE) will attempt to estimate the standard deviation of that irreducible error.\\
        
        It does this by seeing how well your model fits the data!  Convenient, let's calculate the standard deviation of our residuals!  
        
        \begin{equation*}\tag{3.15}
            \text{RSE} = \sqrt{\frac{1}{n-2}\text{RSS}} = \sqrt{\frac{1}{n-2}\sum_{i=1}^n(y_i - \hat y_i)^2}
        \end{equation*}
        RSE tells us about the spread of the observed values from the predicted ones, how far off the model's predictions are, on average..  A lower RSE means the predicted values are closer to the observed ones.\\
        
        RSE is in whatever units the $Y$ variable is in, so it can be hard to understand what it really means.   For that, let's look at...
        \paragraph{$R^2$ Statistic\\}
        $R^2$ also measures the accuracy of the model, but does it as a proportion so the values are always between 0 and 1.
        \begin{equation*}\tag{3.17}
            R^2 = \frac{\text{TSS}-\text{RSS}}{\text{TSS}} = 1 - \frac{\text{RSS}}{\text{TSS}}
        \end{equation*}
        where TSS, the \textbf{total sum of squares} is defined
        \begin{equation*}
            \text{TSS} = \sum_{i=1}^n(y_i - \bar y)^2
        \end{equation*}
        \begin{itemize}
            \tightlist
            \item TSS is the total variance in the response, $Y$.  Think of it as the amount of variability in the response (not dependent on the model)
            \item RSS measures the amount of variability that is left unexplained after performing the regression
            \item TSS - RSS then measures the amount of variability in the response that is explained by performing the regression.
            \item $R^2$ measures the proportion of variability in $Y$ that can be explained using $X$
        \end{itemize}
        An $R^2$ value of $1$ means the model perfectly explains all variability.  An $R^2$ of $0$ means the model explains none of it.  It does just as well as predicting the mean.
        
        \textbf{Should I add correlation?}
        
        \subsection*{3.2 Multiple Linear  Regression}\label{multiple-linear-regression}
        What happens if we have more than one dependent variable?  If we have $p$ input variables, a multiple linear regression model takes the form:
        \begin{equation*}\tag{3.19}
            Y = \beta_0 + \beta_1 X_1 + \beta_2 X_2 + \cdots + \beta_p X_p + \epsilon
         \end{equation*}
         $X_j$ is the $j$th predictor, $\beta_j$ is the corresponding coefficient.  It can be interpreted as the average effect on $Y$ of a one unit increase in $X_j$, given that no other predictor variables change.
        
        \subsubsection*{3.2.1 Estimating the Regression Coefficients}\label{estimating-the-regression-coefficients}
        Our multiple linear regression model will be
        \begin{equation*}\tag{3.21}
            \hat y = \hat \beta_0 + \hat \beta_1 x_1 + \hat \beta_2 x_2 + \cdots + \hat \beta_p x_p.
        \end{equation*}
        Like in simple linear regression, we will choose our $\beta$s in order to minimize the residual sum of squares.
        \begin{align*}
            \text{RSS} &= \sum_{i=1}^n (y_i - \hat y_i)^2\\
                                &= \sum_{i=1}^n (y_i - \hat \beta_0 - \hat \beta_1 x_{i1} - \hat \beta_2 x_{i2} - \cdots - \hat \beta_p x_{ip})^2
        \end{align*}
        \begin{note}
            While not covered by ISLP, in order to make the book more approachable (and not dependent on linear algebra), you will often see multiple linear regression represented in matrix form.
            If we let
            \begin{equation*}
                \boldsymbol{\beta} = \begin{bmatrix*}
                    \hat \beta_0\\
                    \hat \beta_1 \\
                    \hat \beta_2 \\
                    \vdots \\
                    \hat \beta_n
                \end{bmatrix*} \qquad \text{and} \qquad 
                \mathbf{X}_i= \begin{bmatrix*}
                    1 & x_{i1} & x_{i2} & \cdots & x_{in}
                \end{bmatrix*}
            \end{equation*}
            then
            \begin{equation*}
                \hat y_i = \mathbf{X}_i\boldsymbol{\beta}
             \end{equation*}
             
             Note that you will likely see this represented as multiple rows in $\mathbf{X}$ and $\mathbf{Y}$ representing different observations.  I've omitted that to make it easier to understand.
        \end{note}
        
        
        \subsubsection*{3.2.2 Some Important Questions}\label{some-important-questions}
        
        \subsection*{3.3 Other Considerations in the Regression Model}\label{other-considerations-in-the-regression-model}
        
        \subsection*{3.4 The Marketing Plan}\label{the-marketing-plan}
        
        \subsection*{3.5 Comparison of Linear Regression with K-Nearest Neighbors}\label{comparison-of-linear-regression-with-k-nearest-neighbors}
        
    
\end{document}