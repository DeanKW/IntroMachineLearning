  %%%%%%%%%%%%%%%%%%%%%%%%%%%%%%%%%%%%%%%%%%%%%%%%%%%%%%%%%%%%%%%%%%%%%%%%%%%%%%%%%%%%
% Do not alter this block (unless you're familiar with LaTeX
\documentclass{article}
\usepackage[margin=1in]{geometry}
\usepackage[dvipsnames]{xcolor}
\usepackage{tcolorbox}
\usepackage{float}
\usepackage{amsmath,amsthm,amssymb,amsfonts, fancyhdr, color, comment, graphicx, environ,mathtools}
\usepackage{caption}
\usepackage[parfill]{parskip}
\usepackage{mdframed}
\usepackage[shortlabels]{enumitem}
\usepackage{indentfirst}
\usepackage{hyperref}


\hypersetup{
    colorlinks=true,
    linkcolor=blue,
    filecolor=magenta,      
    urlcolor=blue,
}

\newcommand\aug{\fboxsep=-\fboxrule\!\!\!\fbox{\strut}\!\!\!}


\pagestyle{fancy}


\newenvironment{problem}[2][Problem]
{ \begin{mdframed}[backgroundcolor=gray!20] \textbf{#1 #2} \\}
    {  \end{mdframed}}

% Define solution environment
\newenvironment{solution}
{\textit{Answer:}}
{}

%%%%%%%%%%%%
% NOTE
%%%%%%%%%%%%
\makeatletter
\newfloat{info@box}{tbp}{loi}[section]% 1: Name of float environment. 2: Default placement (top, bottom, ...). 3: File extension if written to an aux-file (like toc, lof, lot, loa, ...). 4: Numbering within <section/subsection/...>.
\makeatother
\floatname{info@box}{Infobox}% Adapt caption.

\newenvironment{infobox}[1][]{% Create new environment using info@box and tcolorbox
    %\begin{info@box}%
        \begin{tcolorbox}[colback=red!15!white,%                    background color
            colframe=red!75!black,%                               frame color
            title=Additional information\ifstrempty{#1}{}{: #1}.% title
            ]%
        }{%
        \end{tcolorbox}%
%    \end{info@box}%
}


\newenvironment{note}[1][]{% Create new environment using info@box and tcolorbox
%    \begin{info@box}%
        \begin{tcolorbox}[colback=blue!15!white,%                    background color
            colframe=blue!75!black,%                               frame color
            title=Note\ifstrempty{#1}{}{: #1}% title
            ]%
        }{%
        \end{tcolorbox}%
    %\end{info@box}%
}

\newenvironment{warning}[1][]{% Create new environment using info@box and tcolorbox
    %    \begin{info@box}%
        \begin{tcolorbox}[colback=red!15!white,%                    background color
            colframe=red!75!black,%                               frame color
            title=Warning\ifstrempty{#1}{}{: #1}% title
            ]%
        }{%
        \end{tcolorbox}%
        %\end{info@box}%
    }


\newenvironment{interesting}[1][]{% Create new environment using info@box and tcolorbox
    \begin{info@box}%
        \begin{tcolorbox}[colback=green!15!white,%                    background color
            colframe=green!75!black,%                               frame color
            title=Interesting Tidbit!\ifstrempty{#1}{}{: #1},% title
            ]%
        }{%
        \end{tcolorbox}%
    \end{info@box}%
}

\newenvironment{aside}[1][]{% Create new environment using info@box and tcolorbox
    %    \begin{info@box}%
        \begin{tcolorbox}[colback=yellow!15!white,%                    background color
            colframe=yellow!60!black,%                               frame color
            title=An Aside\ifstrempty{#1}{}{: #1},% title
            ]%
        }{%
        \end{tcolorbox}%
        %\end{info@box}%
    }
%%%%%%%%%%%%
% END NOTE
%%%%%%%%%%%%

\providecommand{\tightlist}{    \setlength{\itemsep}{0pt}\setlength{\parskip}{0pt}}

\renewcommand{\qed}{\quad\qedsymbol}

\DeclareMathOperator*{\argmin}{argmin}
\DeclareMathOperator*{\argmax}{argmin}

\newtheorem*{bookequation}{Equation}

% prevent line break in inline mode
\binoppenalty=\maxdimen
\relpenalty=\maxdimen

\date{\vspace{-5ex}}